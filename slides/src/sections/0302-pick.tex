\begin{frame}{\secname: \subsecname}

Finger $F$ picks a string segment $s$ 

\begin{itemize}
    \item Written as $F(s)$
\end{itemize}

\pause Four variations:
\begin{itemize}[<+(1)->]
    \item $F$ passes \emph{over/under} all intermediate segments
    \item $F$ picks $s$ from \emph{above/below}
    \note<.(1)>[item]{(CAMERA) show all 4 cases on $L1:R1$ with $R5(R1n)$}
\end{itemize}

\pause Examples
\begin{itemize}[<+(1)->]
    \item "$R5$ passes \emph{over} all intermediate segments and picks $Lp$ from \emph{above}" is denoted  as $\overset\Longleftarrow{R5}(\overline{Lp})$
    \item "$R1$ passes \emph{over} all intermediate segments and picks $R5n$ from \emph{below}" is denoted as $\overrightarrow{R1}(\underline{R5n})$
    \note<.(1)>[item]{the first 2 moves makes the star}
    \item "$R4$ passes \emph{below} all intermediate segments and picks $L1n$ from \emph{below}" is denoted as $\underset\Longleftarrow{R4}(\underline{L1n})$
\end{itemize}
\end{frame}
\note[itemize]{
\item have an arrow above the finger means pass \emph{over}, underarrow means pass \emph{under}
\item an underlined segment means pick the segment from below, overline means from above
\item this notations come from a monograph written by a mathematician Tom Storer
\item Storer says that the single arrow is for same hand pick, and right arrow means finger moves away
\item says that double arrow for opposite hand pick, left arrow means right finger picks left segments
\item the direction of the arrows can be determined by the finger and segment, so its redundant, but i'm following Storer's convention
}

\begin{frame}{\subsecname: Examples}
\note<1>[item]{now lets look at how these different pick moves change the linear sequence}
\begin{adjustwidth}{-2em}{-2em}
Starting with \g{figures/pick-before}$\enspace [n]L1[f]:[f]R1[n]$\\
\begin{minipage}{0.55\columnwidth}
\pause $\underleftarrow{R5}(\underline{R1n})$ gives \g[0.3]{figures/pick-under-below}\\
$L1:x_1(o):x_2(o):R1:x_2(u):R5:x_1(u)$\\
\vfill
\pause $\underleftarrow{R5}(\overline{R1n})$ gives \g[0.3]{figures/pick-under-above}\\
$L1:x_1(o):x_2(o):R1:x_2(u):x_3(u):R5:x_3(o):x_1(u)$\\
\end{minipage}
\hfill
\begin{minipage}{0.55\columnwidth}
\pause $\overleftarrow{R5}(\underline{R1n})$ gives \g[0.3]{figures/pick-over-below}\\
$L1:x_1(u):x_2(u):R1:x_2(o):R5:x_1(o)$\\
\vfill
\pause $\overleftarrow{R5}(\overline{R1n})$ gives \g[0.3]{figures/pick-over-above}\\
$L1:x_1(u):x_2(u):R1:x_2(o):x_3(u):R5:x_3(o):x_1(o)$\\
\end{minipage}\\
\pause Observations
\begin{itemize}[<+(1)->]
    \item A pair of crossings for each intermediate string
    \item $\overleftarrow F(s)$ and $\underleftarrow F(s)$ differ by crossing parity
    \item $F(\overline s)$ and $F(\underline s)$ differ by a twist
\end{itemize}
\end{adjustwidth}
\end{frame}

\note[itemize]{
\item in this case, we always insert the new finger $R5$, at where $R1n$ is in the sequence
\item and insert $x_1$ at $L1f$ and $x_2$ at $R1f$ since intermediate
\item when $F$ moves away, the new twist is away, e.g. $L5:R5$ and pick with $R1$
}


\begin{frame}{\subsecname: Construction}
General steps for applying $F(s)$ to a string figure
\begin{itemize}[<+(1)->]
    \item Identify intermediate segments
    \item Insert a pair of crossings for each intermediate segment
    \item Insert $F$ at $s$ with crossings
    \item Add twist if pick from above
\end{itemize}

% write operation
\g[0.27]{figures/star-before}
\uncover<3->{$\mapsto$\g[0.27]{figures/star-pick}}
\uncover<5->{$\mapsto$\g[0.27]{figures/star-pick-twist}}

\note[item]{in this case, suppose i want to apply (WRITE) $\overset\Longleftarrow{R5}(\overline{Lp})$}
\end{frame}
\note[itemize]{
\item this is a visual of what we should expect when we apply the computation, 
\item now we will this without diagrams, working on linear sequences only
}
