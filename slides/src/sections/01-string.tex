\section{String Figures}

\begin{frame}{\secname}
\begin{itemize}
    \item Designs formed from a loop of string
    \item Commonly known as a children's game
\end{itemize}

\pause People have been playing with the string around the world for millennia.

\begin{itemize}[<+(1)->]
    \item  Entertainment during polar nights in the Arctic region
    \item  Storytelling and illustrating scenes from myths and legends
\end{itemize}

\textcolor{red}{todo insert 3 images of strings figures from the encyclopedia: the star, something i can do, something i cant do}
\end{frame}

\note[itemize]{
\item the native inhabitants in the arctic region play string figures for entertainment durong polar nights
\item the indigenous people in New Zealand play string figures for storytelling and illustrating scenes from myths and legends
\item and i play string figures when overleaf takes forever to compile my slides
}


\begin{frame}{A Computational Approach}
String figures take a lot of steps to make

\begin{itemize}[<+(1)->]
    \item  Start with an initial position
    \item  Apply a sequence of moves
    \item  Each move transforms a string figure to another
\end{itemize}

\pause String figures computations

\begin{itemize}[<+(1)->]
    \item  Represent string figures: simple, precise
    \item  Apply moves directly to the representations
\end{itemize}

\pause Motivation

\begin{itemize}[<+(1)->]
    \item Do string figures on paper
    \item Computer simulations \& animations
\end{itemize}

\end{frame}

\note[itemize]{
\item (take steps) for example, to make the star, we start with this initial position, and then apply two simple moves of picking a segment of the string 
\item (mot) clear way of describing how to make string figures
\item (mot) teach computers how to play string figures
\item (record) store in computer, as database
\item simliar to music scores, rubiks cubes
\item one effective way of describing string figures is to draw them, as string diagrams
}