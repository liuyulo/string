\section*{Summary}

\begin{frame}{\secname}
What we covered
\begin{itemize}[<+(1)->]
    \item Representing string figures as linear sequences
    \note[<.(1)>]{we trace the fingers and the crossings}
    \note[<.(1)>]{we can easily store it as a list of symbols for computers to use}
    \item Applying twist and pick to linear sequences
    \note[<.(1)>]{they reflect the movements we do physically on string figures}
    \note[<.(1)>]{they create new crossings}
\end{itemize}

\pause Going deeper

\begin{itemize}[<+(1)->]
    \item More moves 
    \note<.(1)>[item]{one of the simpliest moves that can decrease the number of crossings is drop}
    \note<.(1)>[item]{basically undoing the pick}
    \note<.(1)>[item]{but recognising when crossings can get cancelled is not trivial}
    \note<.(1)>[item]{...}
    \note<.(1)>[item]{when i drop all fingers, i should be able to cancel all crossings, which gives me a unknot}
    \note<.(1)>[item]{and we don't know if we can recognise unknot in polynomial time}
    \item Drawing diagrams from linear sequences
    \note<.(1)>[item]{requires physics that i have zero clue about, Alfredo's code can do and idk how}
    \note<.(1)>[item]{...}
    \note<.(1)>[item]{code demo}
\end{itemize}
\end{frame}